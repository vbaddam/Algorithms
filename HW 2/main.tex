\documentclass[11pt]{article}
\usepackage{amsmath, amssymb, amscd, amsthm, amsfonts}
\usepackage{graphicx}
\usepackage{hyperref}
\usepackage{algorithm}
\usepackage{algorithmic}
\usepackage{amsmath}
\usepackage{algorithm}
\usepackage[noend]{algpseudocode}
\usepackage{mathtools}

\newcommand\Myperm[2][^n]{\prescript{#1\mkern-2.5mu}{}P_{#2}}
\newcommand\Mycomb[2][^n]{\prescript{#1\mkern-0.5mu}{}C_{#2}}





 

\oddsidemargin 0pt
\evensidemargin 0pt
\marginparwidth 40pt
\marginparsep 10pt
\topmargin -20pt
\headsep 10pt
\textheight 8.7in
\textwidth 6.65in
\linespread{1.2}

\title{Home work 2}
\author{Vasanth Reddy Baddam}




%\author{Homework by Vasanth Reddy}
\date{02/13/2020}

\newtheorem{theorem}{Theorem}
\newtheorem{lemma}[theorem]{Lemma}
\newtheorem{conjecture}[theorem]{Conjecture}

\newcommand{\rr}{\mathbb{R}}

\newcommand{\al}{\alpha}
\DeclareMathOperator{\conv}{conv}
\DeclareMathOperator{\aff}{aff}

\begin{document}

\maketitle
I pledge that this test/assignment has been completed in compliance with the Graduate Honor Code and
that I have neither given nor received any unauthorized aid on this test/assignment\\
\textbf{Name: } Vasanth Reddy Baddam\\
\textbf{Signature: } VB\\
\hline
\vspace{5mm}
\textbf{Q1.}\\
p = [5,10,3,12,5,50,6]\\
n = length(p) - 1\\
Let's assume two matrices $m$ and $s$ of size n*n\\
Let's say we are going through the chain of matrices of length\\
for l = 1, Assuming that we only have one matrix on the multiplication. Then the cost for that multiplication would be zero.\\
So we replace the elements of matrix $m$ ($m[i][i]$) with zero\\
\textbf{for l = 2}\\
$$m[1,2] = m[1,1] + m[2,2] + p[0]*p[1]*p[2]$$
$$m[1,2] = 0 + 0 + 5*10*3$$
$$m[1.2] = 150$$
$$m[2,3] = m[2,2] + m[3,3] + p[2]*p[3]*p[4]$$
$$m[2,3] = 0 + 0 + 10*3*12$$
$$m[2.3] = 360$$
$$m[3,4] = m[3,3] + m[4,4] + p[3]*p[4]*p[5]$$
$$m[3,4] = 0 + 0 + 3*12*5$$
$$m[3,4] = 180$$
$$m[4,5] = m[4,4] + m[5,5]+ p[5]*p[4]*p[6]$$
$$m[4,5] = 0 + 0 + 12*5*50$$
$$m[4,5] = 3000$$
$$m[5,6] = m[5,5]+ m[6,6] + p[5]*p[6]*p[7]$$
$$m[5,6] = 0 + 0 + 5*50*6$$
$$m[5,6] = 1500$$
\textbf{for l = 2}
$$m[1,3] = min{(m[1,1] + m[2,3] + p[0]*p[1]*p[3]), (m[1,2] + m[3,3] + p[0]*p[2]*p[3])}$$
$$m[1,3] = min{(0 + 360 + 600), (150 + 0 + 180)}$$
$$m[1,3] = min{960, 330}$$
$$m[1,3] = 330$$
Using the recursive value from the above findings for $l = 2$ from $l = 1$\\
Similarly filling the matrix $m$, we get \\
\begin{center}
 \begin{tabular}{||c c c c c c c||} 
 \hline
 i/j & 1 & 2 & 3 & 4 & 5 & 6 \\ [0.5ex] 
 \hline\hline
 1 & 0 & 150 & 330 & 405 & 1655 & 2010 \\ 
 \hline 
 2 & 0 & 0   & 360 & 330 & 2430 & 1950 \\
 \hline
 3 & 0 & 0   & 0   & 180 & 930  & 1770 \\
 \hline
 4 & 0 & 0   & 0   & 0   & 3000 & 1860 \\
 \hline
 5 & 0 & 0   & 0   & 0   & 0    & 1500 \\
 \hline 
 6 & 0 & 0   & 0   & 0   & 0    & 0\\[1ex] 
 \hline
\end{tabular}
\end{center}
K table is given by \\
\begin{center}
 \begin{tabular}{||c c c c c c c||} 
 \hline
 i/j & 1 & 2 & 3 & 4 & 5 & 6 \\ [0.5ex] 
 \hline\hline
 1 & 0 & 1 & 2 & 2 & 4 & 2 \\ 
 \hline 
 2 & 0 & 0   & 2 & 2 & 2 & 2 \\
 \hline
 3 & 0 & 0   & 0   & 3 & 4  & 4 \\
 \hline
 4 & 0 & 0   & 0   & 0   & 4 & 4 \\
 \hline
 5 & 0 & 0   & 0   & 0   & 0    & 5\\
 \hline 
 6 & 0 & 0   & 0   & 0   & 0    & 0\\[1ex] 
 \hline
\end{tabular}
\end{center}

\hline
\vspace{5mm}
\textbf{Q2.}\\

\begin{center}
 \begin{tabular}{||c c c c c c c c c c||} 
 \hline
 i/j &0& S & P & A & N & K & I & N & G \\ [0.5ex] 
 \hline\hline
 0 & 0 & 0 & 0 & 0 & 0 & 0&0&0 &0\\ 
 \hline 
 A & 0 & 0   & 0 & 1 & 1 & 1&1&1&1 \\
 \hline
 M & 0 & 0   & 0 & 1 & 1 & 1&1&1&1  \\
 \hline
 P & 0 & 0   & 1 & 1 & 1 & 1&1&1&1  \\
 \hline
 U & 0 & 0   & 1 & 1 & 1 & 1&1&1&1 \\
 \hline 
 T & 0 & 0   & 1 & 1 & 1 & 1&1&1&1 \\
 \hline
 A&0&0&1&2&2&2&2&2&2\\
 \hline
 T&0&0&1&2&2&2&2&2&2\\
 \hline
 I&0&0&1&2&2&2&3&3&3\\
 \hline
 O&0&0&1&2&2&2&3&3&3\\
 \hline
 N&0&0&1&2&3&3&4&4&4\\[0.2ex]
 \hline
\end{tabular}
\end{center}
Direction table is given by:\\
\begin{center}
 \begin{tabular}{||c c c c c c c c c c||} 
 \hline
 
 i/j &0& S & P & A & N & K & I & N & G \\ [0.5ex] 
 
 0 & 0 & 0 & 0 & 0 & 0 & 0&0&0 &0\\ 
 
 A & 0 & $\uparrow$   & $\uparrow$ & $\uparrow$ & $\nwarrow$ &$\leftarrow$&$\leftarrow$&$\leftarrow$&$\leftarrow$ \\
 
 M & 0 & $\uparrow$ & $\uparrow$ & $\uparrow$ & $\uparrow$ & $\uparrow$&$\uparrow$&$\uparrow$&$\uparrow$ \\ 
 
 P & 0 & $\uparrow$   & $\nwarrow$ & $\uparrow$& $\uparrow$& $\uparrow$&$\uparrow$&$\uparrow$&$\uparrow$\\ 
 
 U & 0 & $\uparrow$   & $\uparrow$ & $\uparrow$ &$\uparrow$ & $\uparrow$&$\uparrow$&$\uparrow$&$\uparrow$\\
 
 T & 0 & $\uparrow$   & $\uparrow$ & $\uparrow$ &$\uparrow$ & $\uparrow$&$\uparrow$&$\uparrow$&$\uparrow$\\
 
 A& 0 & $\uparrow$   & $\uparrow$ & $\nwarrow$ &$\leftarrow$ & $\leftarrow$&$\leftarrow$&$\leftarrow$&$\leftarrow$\\
 T& 0 & $\uparrow$   & $\uparrow$ & $\uparrow$ &$\uparrow$ & $\uparrow$&$\uparrow$&$\uparrow$&$\uparrow$\\
 
 I& 0 & $\uparrow$   & $\uparrow$ & $\uparrow$ &$\uparrow$ & $\uparrow$&$\nwarrow$&$\leftarrow$&$\leftarrow$\\

 O& 0 & $\uparrow$   & $\uparrow$ & $\uparrow$ &$\uparrow$ & $\uparrow$&$\uparrow$&$\uparrow$&$\uparrow$\\
 
 N& 0 & $\uparrow$   & $\uparrow$ & $\uparrow$ &$\nwarrow$ & $\leftarrow$&$\uparrow$&$\nwarrow$&$\leftarrow$\\[0.2ex]
 \hline
\end{tabular}
\end{center}
Following the diagonal arrows we get the longest common sub sequence is PAIN\\


\hline
\vspace{5mm}
\textbf{Q4.}\\
\hspace*{2cm} Given that the splits in quick sort are in the proportion of $1-\alpha$ and $\alpha$\\
As we seen that minimum depth always take the smaller part of the split i.e.. $\alpha$. Where as the maximum depth always takes the larger split i.e.. $1-\alpha$. As the number of passes increases, $\alpha$ and $1-\alpha$ changes to $\alpha^a$ and $1-\alpha^b$ respectively\\
At the end of the tree with one node remaining for minimum depth, then
$$n*\alpha^a = 1$$
$$\alpha^a = \frac{1}{n}$$
applying log on both sides, \\
$$a*\log(\alpha) = -\log(n)$$
$$a = -\frac{\log(n)}{\og(\alpha)}$$
At the end of the tree with one node remaining for maximum depth, then
$$n*(1-\alpha)^{b}$$
$$(1-\alpha)^{b} = \frac{1}{n}$$
applying log on both sides, \\
$$b*\log(1-\alpha) = -\log(n)$$
$$b = -\frac{\log(n)}{\og(1-\alpha)}$$
\hline
\vspace{5mm}
\textbf{Q5.}\\
As we know, the condition i.e ..\\
$$n = \frac{D*E}{P + E}$$
where D is Maximum number of elements can be in an array\\
E is Space for data value\\
P is Space for pointer\\
\textbf{a).}\\
space of an array = number of elements can hold*size of data field\\
$$\text{size of an array} = 20*8$$ bytes\\
$$\text{size of an array} = 160$$ bytes\\
space for an one node in linked list requires  = $8 + 4$ bytes\\
Condition to an linked list when:\\
$$n*12 <= 160$$
$$n <= 13.33$$
for $n <= 13$, linked list needs space less than array\\
\textbf{b).}\\
space of an array = number of elements can hold*size of data field\\
$$\text{size of an array} = 30*2$$ bytes\\
$$\text{size of an array} = 60$$ bytes\\
space for an one node in linked list requires  = $4 + 2$ bytes\\
Condition to an linked list when:\\
$$n*6 <= 60$$
for $n < 10$, linked list needs space less than array\\
\textbf{c).}\\
space of an array = number of elements can hold*size of data field\\
$$\text{size of an array} = 30*1$$ bytes\\
$$\text{size of an array} = 30$$ bytes\\
space for an one node in linked list requires  = $1 + 4$ bytes\\
Condition to an linked list when:\\
$$n*5 <= 30$$
for $n <6$, linked list needs space less than array\\
\textbf{d).}\\
space of an array = number of elements can hold*size of data field\\
$$\text{size of an array} = 40*32$$ bytes\\
$$\text{size of an array} = 1280$$ bytes\\
space for an one node in linked list requires  = $32 + 4$ bytes\\
Condition to an linked list when:\\
$$n*36 <= 1280$$
$$n < 35.55$$
for $n < 35$, linked list needs space less than array\\
\hline
\vspace{5mm}
\textbf{Q3.}\\
\begin{algorithm}
\caption{Rod-Cutting w/Cost}\label{euclid}
\begin{algorithmic}[1]
\Procedure{\textsc{Rod-Cutting}(\textit{p,n,c})}{}
\State let r[0,1,..n] be a new array)
\State r[0] = 0
\For{\texttt{j = 1 to n}}
        \State \texttt{q = p[j]}\\
        \For{\texttt{i = 1 to j -1 }}
        \State \texttt{q = max({q, p[i] + r[j-i] - c})}
      \EndFor
      r[j] = q
      \EndFor
      \State return(r[n])
\EndProcedure
\end{algorithmic}
\end{algorithm}
\end{document}

